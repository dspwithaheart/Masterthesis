% Seitenränder
\usepackage{geometry}

% Zeilenabstand
\usepackage{setspace}

% Koma Script - Seitenlayout
\usepackage[
    automark, 		% Kapitelangaben in Kopfzeile automatisch erstellen
	headsepline, 	% Trennlinie unter Kopfzeile
	ilines 			% Trennlinie linksbündig ausrichten
]
{scrlayer-scrpage}

% Koma Script - Inhaltsverzeichnis
\usepackage{tocbasic}

% Silbentrennung
% \providehyphenmins{ngerman}{44}
\providehyphenmins{english}{44}

% Sprachpaket
% \usepackage[ngerman]{babel}
\usepackage[english]{babel}

% Zum einbinden anderer PDF Dokumente
\usepackage{pdfpages}

% Generiert sinnlosen Text
\usepackage{lipsum}

% Hübschere Tabellen
\usepackage{tabularx}
\usepackage{longtable}
\usepackage{booktabs}
% \usepackage[table]{xcolor}  

% Bilder und Tabellen an exakt dieser Position ( [H] ):
\usepackage{float} 		

% Sub-figures 
\usepackage{caption}
\usepackage{subcaption}

% DVD-Verzeichnis darstellen
\usepackage{dirtree}

% Listings darstellen
\usepackage{listings}
\usepackage{lstautogobble}

% Grafiken einbinden
\usepackage{graphicx}
% Biber (Biblography)
\usepackage[
	style=numeric, % Loads the bibliography and the citation style 
	% bibstyle=alphabetic, % load a bibliography style
	% citestyle=alphabetic, % load a citatio style
	sorting=none,
	natbib=true, % define natbib compatible cite commands
	%%--- Backend --- --- ---
	backend=biber,   % (bibtex, biber)
	bibwarn=true,     %
	texencoding=auto, % auto-detect the input encoding
]{biblatex}  
% Other options:
%  style=numeric, % 
%  style=numeric-comp,    % [1-3, 7, 8]
%  style=numeric-verb,    % [2]; [5]; [6]
%  style=alphabetic,      % [Doe92; Doe95; Jon98]
%  style=alphabetic-verb, % [Doe92]; [Doe95]; [Jon98]
%  style=authoryear,      % Doe 1995a; Doe 1995b; Jones 1998
%  style=authoryear-comp, % Doe 1992, 1995a,b; Jones 1998
%  style=authoryear-ibid,
%  style=authoryear-icomp,
%  style=authortitle,
%  style=authortitle-comp,
%  style=authortitle-ibid,
%  style=authortitle-icomp,
%  style=authortitle-terse,
%  style=authortitle-tcomp,
%  style=authortitle-ticomp,

% UML und Sequenzdiagramme
\usepackage{pgf-umlsd}


% Für schöne, selbstgezeichnete graphiken: tikz
\usepackage{tikz}
\usetikzlibrary{patterns}
\usetikzlibrary{positioning}
\usetikzlibrary{shapes,arrows}
\usetikzlibrary{arrows.meta}
\usetikzlibrary{shapes.geometric, arrows}
\usetikzlibrary{calc}
\usetikzlibrary{decorations.pathreplacing}


\usepackage{fontawesome}






% Hyperref für Verlinkungen 
% ACHTUNG: Als letztes Paket ! (Mit ausnahme von Glossaries)
% PDF-Optionen -----------------------------------------------------------------
\usepackage[
	bookmarks,
	bookmarksopen=true,
	colorlinks=true,
	% diese Farbdefinitionen zeichnen Links im PDF farblich aus
	% linkcolor=red, % einfache interne Verknüfungen
	% anchorcolor=black,% Ankertext
	% citecolor=blue, % Verweise auf Literaturverzeichniseinträge im Text
	% filecolor=magenta, % Verkrüfungen, die lokale Dateien öffnen
	% menucolor=red, % Acrobat-Menüpunkte
	% urlcolor=cyan, 
	% diese Farbdefinitionen sollten für den Druck verwendet werden (alles schwarz)
	linkcolor=black, % einfache interne Verknüpfungen
	anchorcolor=black, % Ankertext
	citecolor=black, % Verweise auf Literaturverzeichniseinträge im Text
	filecolor=black, % Verknüfungen, die lokale Dateien öffnen
	menucolor=black, % Acrobat-Menüpunkte
	urlcolor=black, 
	%pagebackref,
	plainpages=false, % zur korrekten Erstellung der Bookmarks
	pdfpagelabels, % zur korrekten Erstellung der Bookmarks
	hypertexnames=false, % zur korrekten Erstellung der Bookmarks
	%linktocpage % Seitenzahlen anstatt Text im Inhaltsverzeichnis verlinken
	linktoc=all
]{hyperref}

\hypersetup{
	pdftitle={\titel},
	pdfauthor={\autor},
	pdfcreator={\autor},
	pdfsubject={\titel},
	pdfkeywords={\titel},
}

\usepackage{scrextend}
\addtokomafont{labelinglabel}{\sffamily}

% Glossar und Abkürzungsverzeichnis, laut doku nach Hyperref
\usepackage[
	acronym, 
	nonumberlist, 
	toc,
	numberline,
	nopostdot,
	nogroupskip
]{glossaries}


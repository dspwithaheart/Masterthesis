% Seitenränder
\geometry{
	paper=a4paper,
	right=20mm,		
	top=20mm,
	left=30mm,
	bottom=30mm
}

% Zeilenabstand 1,5 Zeilen
\singlespacing
%\spacing{1.21}
%\onehalfspacing
%\doublespacing


% Minimal mehr platz nach einem punkt. <-
\frenchspacing

% Schusterjungen und Hurenkinder vermeiden
\clubpenalty = 10000
\widowpenalty = 10000 
\displaywidowpenalty = 10000


% Style der listings
\definecolor{RoyalBlue}{cmyk}{1, 0.50, 0, 0}
\definecolor{codegreen}{rgb}{0,0.6,0}
\definecolor{codegray}{rgb}{0.5,0.5,0.5}
\definecolor{codepurple}{rgb}{0.58,0,0.82}
\definecolor{backcolour}{rgb}{0.95,0.95,0.92}
\lstset{
	basicstyle=\ttfamily,
	numberstyle=\tiny\color{gray},
	keywordstyle=\color{RoyalBlue},
	commentstyle=\color{green}\ttfamily,
	stringstyle=\color{mauve},
	autogobble=true,
	breaklines=true,
	breakatwhitespace=true,
	captionpos = b,
	frame=single, % box um das listing
	numbers=left, %nummerierung um das Listing
	% Nummerierung innerhalb der box
	framexleftmargin=5.5mm,
	xleftmargin=19pt,
	xrightmargin=3.4pt,
} 


%Ligaturen verhindern:
%\DisableLigatures{}
%Was sind ligaturen ?:
%ff und fi ragen ineinander, und auch andere wilde Buchstabenkombinationen


% Kopf und Fußzeile
\pagestyle{scrheadings}
% Kopf- und Fußzeile auch auf Kapitelanfangsseiten
\renewcommand*{\chapterpagestyle}{scrheadings}
% Schriftform der Kopfzeile
\renewcommand{\headfont}{\normalfont}

% Höhe der Kopfzeile
\setlength{\headheight}{14mm}
% Kopfzeile
\ihead{\headmark}
\chead{}
%\ohead{\includegraphics[scale=0.15, trim={2.4cm 24.5cm 2.6cm 2.6cm}, clip]{Bilder/SIEMENS-Logo.pdf}}
\ohead{\includegraphics[scale=0.08]{Bilder/Siemens_Logo.pdf}}
%\ohead{\art} % Masterarbeit statt SIEMENS in der Kopfzeile

% Fußzeile
% Leer, da sich das Layout ändert
\cfoot{}
\ifoot{}
\ofoot{}

% Layout Inhaltsverzeichnis
\setcounter{tocdepth}{2}  % Inhaltsverzeichnis nur bis 2 Unterelemente auflisten
\setcounter{secnumdepth}{3} % Numerierungstiefe für Inhaltsverzeichnis

\DeclareTOCStyleEntries[
% numwidth=1em,					% Abstand Zahl - Überschrift
rightindent=4em,				% Rand nach Links. Relevant für Überschriften mit Zeilenumbruch im Inhaltsverzeichnis
pagenumberbox=\pagenumberbox
]
{tocline}{chapter, section, subsection, subsubsection, paragraph, subparagraph, figure, table}
\newcommand*\pagenumberbox[1]{\mbox{\hspace{2pt}#1}} 

%Höhe von Kapiteln im Text					2.3
%\renewcommand*{\chapterheadstartvskip}{\vspace*{1.0\baselineskip}}% Abstand einstellen
\renewcommand*{\chapterheadstartvskip}{\vspace*{1\baselineskip}}% Abstand einstellen

% Eigenes Design für Abkürzungsverzeichnis
\newglossarystyle{mylong}{%
	\renewenvironment{theglossary}%
	{\begin{longtable}[l]{llp{\glsdescwidth}p{\glspagelistwidth}}}%
		{\end{longtable}}%
	\renewcommand*{\glossaryheader}{}%
	\renewcommand*{\glsgroupheading}[1]{}%
	\renewcommand{\glossentry}[2]{%
		\textsf{\textbf{\glsentryitem{##1}\glstarget{##1}{\glossentryname{##1}}}} &		\glossentrysymbol{##1} &		\glossentrydesc{##1} &		##2\tabularnewline
	}%
	\renewcommand{\subglossentry}[3]{%
		&
		\glssubentryitem{##2}%
		\glossentrysymbol{##2} &
		\glstarget{##2}{\strut}\glossentrydesc{##2} & ##3\tabularnewline
	}%
	\renewcommand*{\glsgroupskip}{%
		\ifglsnogroupskip\else & & &\tabularnewline\fi}%
}

% Abstand zwischen Bild und Bildunterschrift
% \setlength\abovecaptionskip{-20pt}
%Vektorbasierte Standardschrift
\renewcommand{\familydefault}{\sfdefault} 	% %Defaultschrift ohne Serifen (modernen)
\usepackage{sansmath}						% %Matheschrift ohne Serifen
\usepackage{microtype}						% %Verbesserte Textverteilung in Zeile

\lstdefinelanguage{JavaScript}{
  keywords={typeof, new, true, false, catch, function, return, null, catch, switch, var, if, in, while, do, else, case, break},
  keywordstyle=\color{blue}\bfseries,
  ndkeywords={class, export, boolean, throw, implements, import, this, import},
  ndkeywordstyle=\color{codepurple}\bfseries,
  identifierstyle=\color{black},
  sensitive=false,
  comment=[l]{//},
  morecomment=[s]{/*}{*/},
  commentstyle=\color{purple}\ttfamily,
  stringstyle=\color{red}\ttfamily,
  morestring=[b]',
  morestring=[b]"
}

\lstdefinelanguage{Dockerfile}
{
  morekeywords={FROM, RUN, CMD, LABEL, MAINTAINER, EXPOSE, ENV, ADD, COPY,
    ENTRYPOINT, VOLUME, USER, WORKDIR, ARG, ONBUILD, STOPSIGNAL, HEALTHCHECK,
    SHELL},
  morecomment=[l]{\#},
  morestring=[b]"
}

% Docker language
\lstdefinelanguage{docker}{
	keywords={FROM, RUN, COPY, ADD, ENTRYPOINT, CMD,  ENV, ARG, WORKDIR, EXPOSE, LABEL, USER, VOLUME, STOPSIGNAL, ONBUILD, MAINTAINER},
	keywordstyle=\color{blue}\bfseries,
	identifierstyle=\color{black},
	sensitive=false,
	comment=[l]{\#},
	commentstyle=\color{purple}\ttfamily,
	stringstyle=\color{red}\ttfamily,
	morestring=[b]',
	morestring=[b]"
}

% Dockercompose
\lstdefinelanguage{docker-compose}{
	keywords={version, volumes, name, driver, ipam, config, subnet, gateway, services, image, environment, ports, links, restart, build, command, mem\_limit, networks, privileged, group\_add, ipc, network\_mode},
	sensitive=false,
	comment=[l]{\#},
	morestring=[b]',
	morestring=[b]"
}

% Ros2 message
\lstdefinelanguage{msg}{
	keywords={bool, byte, char, float32, float64, int8, uint8, int16, uint16, int32, uint32, int64, uint64, string, wstring},
	sensitive=false,
	comment=[l]{\#},
	morestring=[b]',
	morestring=[b]"
}

% Ros2 service
\lstdefinelanguage{service}{
	keywords={bool, byte, char, float32, float64, int8, uint8, int16, uint16, int32, uint32, int64, uint64, string, wstring},
	sensitive=false,
	comment=[l]{\#},
	morestring=[b]',
	morestring=[b]"
}

% Ros2 action
\lstdefinelanguage{action}{
	keywords={bool, byte, char, float32, float64, int8, uint8, int16, uint16, int32, uint32, int64, uint64, string, wstring},
	sensitive=false,
	comment=[l]{\#},
	morestring=[b]',
	morestring=[b]"
}

% C++

\definecolor{dkgreen}{rgb}{0,0.6,0}
\definecolor{dred}{rgb}{0.545,0,0}
\definecolor{dblue}{rgb}{0,0,0.545}
\definecolor{lgrey}{rgb}{0.9,0.9,0.9}
\definecolor{gray}{rgb}{0.4,0.4,0.4}
\definecolor{darkblue}{rgb}{0.0,0.0,0.6}
\lstdefinelanguage{cpp}{
      backgroundcolor=\color{lgrey},  
      basicstyle=\footnotesize \ttfamily \color{black} \bfseries,   
      breakatwhitespace=false,       
      breaklines=true,               
      captionpos=b,                   
      commentstyle=\color{dkgreen},   
      deletekeywords={...},          
      escapeinside={\%*}{*)},                  
      frame=single,                  
      language=C++,                
      keywordstyle=\color{purple},  
      morekeywords={BRIEFDescriptorConfig,string,TiXmlNode,DetectorDescriptorConfigContainer,istringstream,cerr,exit}, 
      identifierstyle=\color{black},
      stringstyle=\color{blue},      
      numbers=left,                 
      numbersep=5pt,                  
      numberstyle=\tiny\color{black}, 
      rulecolor=\color{black},        
      showspaces=false,               
      showstringspaces=false,        
      showtabs=false,                
      stepnumber=1,                   
      tabsize=5,                     
      title=\lstname,                 
    }

\definecolor{eclipseStrings}{RGB}{42,0.0,255}
\definecolor{eclipseKeywords}{RGB}{127,0,85}
\colorlet{numb}{magenta!60!black}

	\lstdefinelanguage{json}{
    basicstyle=\normalfont\ttfamily,
    commentstyle=\color{eclipseStrings}, % style of comment
    stringstyle=\color{eclipseKeywords}, % style of strings
    numbers=left,
    numberstyle=\tiny\color{black}, 
    stepnumber=1,
    numbersep=8pt,
    showstringspaces=false,
    breaklines=true,
    frame=single,
    backgroundcolor=\color{lgrey}, %only if you like
    string=[s]{"}{"},
    comment=[l]{:\ "},
    morecomment=[l]{:"},
    literate=
        *{0}{{{\color{numb}0}}}{1}
         {1}{{{\color{numb}1}}}{1}
         {2}{{{\color{numb}2}}}{1}
         {3}{{{\color{numb}3}}}{1}
         {4}{{{\color{numb}4}}}{1}
         {5}{{{\color{numb}5}}}{1}
         {6}{{{\color{numb}6}}}{1}
         {7}{{{\color{numb}7}}}{1}
         {8}{{{\color{numb}8}}}{1}
         {9}{{{\color{numb}9}}}{1}
}
\chapter{Approach}
\label{Approach}
	Im Rahmen dieser Masterarbeit werden mehrere Kommunikationswege über verschiedene Geräte verwendet.
	So ist es ein Ziel, das Framework ROS2 zu verwenden.
		
	\section{Testsetup}
	\label{Voruntersuchungen:Testsetup}
		Mithilfe einer Testumgebung können verschiedene Konfigurationen getestet und entwickelt werden.
		Notwendig ist, dass das Testsetup entscheidende Merkmale des Netzwerkes in der realen Umgebung beibehält.
		So ist die Aufteilung der Teilnehmer in zwei verschiedene Netzwerke wichtig, da die zu entwickelnde Edge App auf die Netzwerkeinstellung "host" verzichten soll.
		Ebenfalls wird dadurch eine sehr viel höhere Flexibilität der Netzwerkstruktur ermöglicht. 
				
		\subsection{Host Only}
		\label{Voruntersuchungen:Testsetup:HostOnly}
			Das Erste der beiden Testnetzwerke erstellt drei Virtuelle Maschinen: VM A, VM B und VM Router.
			Die Virtuelle Maschine VM A ist dem Subnetz 192.168.64.0/24 zugeordnet. VM B dem Subnetz 192.168.80.0/24.
			Als Brücke beider Subnetze dient VM Router, welche eine Verbindung zu beiden Netzwerken aufbaut und sämtliche Pakete an das jeweils andere Netz schickt.
			
		
	
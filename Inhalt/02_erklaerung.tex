
% \chapter*{Sperrvermerk}
% \label{Sperrvermerk}
% Die vorliegende Arbeit \glqq \textbf{Eine Methode zur Evaluierung und Implementierung eines Datenbankkonzeptes zur Prozessoptimierung einer SW-Testdatenauswertung}\grqq\ beinhaltet interne vertrauliche Informationen der AKKA DSO Ingolstadt GmbH. Die Weitergabe des Inhalts der Arbeit im Gesamten oder in Teilen sowie das Anfertigen von Kopien oder Abschriften -- auch in digitaler Form -- sind grundsätzlich untersagt. Ausnahmen bedürfen der schriftlichen Genehmigung der AKKA DSO Ingolstadt GmbH.
% \addcontentsline{toc}{chapter}{Sperrvermerk}

\chapter*{Eidesstattliche Erklärung}
\label{erklaerung}

Hiermit versichere ich, die vorliegende Abschlussarbeit selbstständig und nur unter Verwendung der von mir angegebenen Quellen und Hilfsmittel verfasst zu haben. Sowohl inhaltlich als auch wörtlich entnommene Inhalte wurden als solche kenntlich gemacht. Die Arbeit hat in dieser oder vergleichbarer Form noch keinem anderem Prüfungsgremium vorgelegen. \\
\\[1.5cm]
Datum:	\hrulefill\enspace Unterschrift: \hrulefill
\\[3.5cm]
% \addcontentsline{toc}{chapter}{Eidesstattliche Eklärung}

% \chapter*{Danksagung}
% \label{Danksagung}
% Danke an alle

% % \addcontentsline{toc}{chapter}{Danksagung}

% \chapter*{Zusammenfassung}
% \label{Zusammenfassung}
% Die AKKA DSO Ingolstadt GmbH beschäftigt sich mit der Entwicklung und Erprobung von Software für verschiedene Fahrzeugsteuergeräte, Fahrzeugfunktionen und Fahrerassistenzsysteme. Dazu werden HiL-Simulatoren (Hardware in the Loop) eingesetzt. Im Labor werden Teile des Fahrzeugs oder gegebenenfalls das gesamte Fahrzeug so simuliert, dass ein Fahrzeugsteuergerät und andere Fahrzeugfunktionen unter verschiedenen Betriebssituationen getestet werden können. Dies ermöglicht die Automatisierung von Fahrzeugtests und beschleunigt den gesamten Entwicklungsprozess. Während des Testprozesses entstehen riesige Datenvolumen, die für die Analyse der einzelnen Tests und für die Verfolgung des Entwicklungsfortschritts notwendig sind. Es scheint ein Bedarf für eine Anwendung mit zentraler Datenbank zu bestehen, die einen einfachen Zugriff auf die Test-Daten ermöglicht und den Verlauf des Entwicklungsprozesses leichter nachvollziehbar macht. Diese Anwendung muss in der gesamten Abteilung von allen beteiligten Mitgliedern benutzbar sein. Diese Bachelorarbeit beschreibt die Konzipierung und eine Methode zur prototypischen Entwicklung dieses Testdatenauswertungswerkzeugs.

% Der Schwerpunkt dieser Arbeit liegt allerdings auf dem Entwurf und der Implementierung eines webbasierten Datenverwaltungsystems. Die verwendete Softwarearchitektur, der Entwicklungsprozess einer Webanwendung, die eingesetzten Frameworks und das zugrundeliegende Datenbanksystem werden im Detail beschrieben. Die zum Verständnis einer Webanwendung notwendigen Grundlagen werden im Kapitel \ref{sec:grundlagen} erläutert. In der Entwurfsphase werden die Anforderungen an das System ermittelt und das Anwendungskonzept unter Berücksichtigung der erarbeiteten Anforderungen erstellt. Für die Implementierung der Webanwendung werden die Programmiersprachen Python und Javascript verwendet.

% \addcontentsline{toc}{chapter}{Zusammenfassung}

% \chapter*{Abstract}
% \label{Abstract}
% AKKA DSO Ingolstadt GmbH is engaged in the development and test of software for various vehicle control units, vehicle functions and driver assistance systems. HiL simulators are used for this purpose. In the laboratory, parts of the vehicle or, if necessary, the entire vehicle is simulated in such a way that a vehicle control unit and other vehicle functions can be tested under various operating situations. This enables the automation of vehicle tests and accelerates the entire development process. During the test process, large volumes of data are generated, which are necessary for the analysis of the individual tests and for tracking the progress of the development. There seems to be a need for an application with a central database, which allows easy access to the test data and makes it easier to track the progress of the development process. This application must be usable throughout the department by all involved members. This bachelor thesis describes the conception and a method for the prototypical development of this test data evaluation tool.

% However, the focus of this work is on the design and implementation of a web-based data management system. The chosen software architecture, development process of a web application, used frameworks and underlying database system are described in detail. The basics necessary for understanding a web application are explained in the chapter \ref{sec:grundlagen}. In the design phase, the requirements for the system are determined and the application concept is developed taking the determined requirements into consideration. The programming languages Python and Javascript are used to implement the web application.

% \addcontentsline{toc}{chapter}{Abstract}
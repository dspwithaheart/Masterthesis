\chapter{Discussion}
\label{Discussion}

Introduction of ROS2 Applications with managed lifecycle alone provides huge improvement to the overall operation of a mobile robot system. It provides the potential for reducing operational and maintenance costs and improves the compatibility of an application with other future applications. A ROS2 Application with managed lifecycle opens up several potential use cases and advantages during the operation of a mobile robot.
\\

It enables dynamic reconfiguration of various operating parameters of component nodes (applications) within a robot system. Configurations for different hardware sensors and actuators of a robot can be adjusted remotely and individually just by switching the state of an application (node). Other applications (node) can continue performing their designated functions while others are being reconfigured depending on the current conditions. For example, the input sensitivity of a joystick can be changed or a damaged modular component (camera, laser scanner, etc.) can be changed and reconfigured while the rest of the system keeps running. It also allows respawning or restarting of any independent subsystem on the fly. If a sub-application (node) depends on another sub-application (node) it can switch to an idle state and wait for another application to be fully functional without shutting itself down completely. This localizes any failure to a sub-application which can then be diagnosed and repaired individually. This also enables active status monitoring of various subsystems for early and efficient maintenance. Managed lifecycle also improves the error logging process of various subsystem applications. This in turn streamlines the debugging process by providing a detailed and more descriptive remote logging system. This adds a level of efficiency to the system and makes the overall operation smoother. It also lays the foundation for achieving complete modularity.
\\

The objective of modern Application Development is for many developers to work simultaneously on different functions of the same application. If an organization is set up so that all functionalities are to be merged in a single instance, the consequential work can be tedious, manual, and time-consuming. When a developer working in isolation makes a change to an application, it may conflict with other changes made simultaneously by other developers. This problem is compounded if each developer has customized their local development environment. One of the objectives of this thesis is to improve the overall stability and compatibility of the applications developed by the organization. Which includes the improvement and refinement of the development and operational environment of the developed applications. To achieve complete automation in the software integration and deployment of ROS2-based mobile robots steps relating to building a CI/CD pipeline are necessary. 
\\

\textit{
CI/CD is a method for deploying applications continually to end users by introducing automation into the application development stages. The core concepts associated with CI/CD are continuous integration, continuous delivery, and continuous deployment. CI/CD provides a solution to the challenges posed to development and operational teams through the integration of new code. In particular, it introduces end-to-end automation and constant monitoring during the entire application lifecycle, from the integration and testing phase through to the deployment phase.\cite*{cicdOverwiew}}
\\

This is achieved by containerizing the ROS2 Applications in Docker containers, which are then converted to IE Apps that can be uploaded to IE Hub for further deployment. Above mentioned design and implementation provide the following advantages:


\begin{itemize}
	\item 	Allows package updates to be installed without shutting down the entire system. (Modularity/ CI/CD)
	\item 	Streamlines the debugging and error logging process of subsystem applications.
	\item 	Simplifies the installation of additional modular components on an existing mobile robot or the sharing of modular components among a group of robots.
	\item 	Enables active status monitoring of various subsystems for early and efficient maintenance.
	\item 	Prevents total system failure by loosely coupling different subsystems and avoiding a monolithic architecture.
	\item 	Provides developers with the freedom to choose from different libraries and frameworks which do not have to be version compatible.
	\item 	Supports the development of an autonomously operating and updating fleet of mobile robots.
\end{itemize}

\chapter{Conclusion and Future Work}
\label{: Conclusion and Future Work}
	This thesis demonstrates the development process of a ROS2 based Robot Applications with inbuilt Lifecycle. Furthermore the application is packed into a Docker environment which further simplifies the development process and helps integrate custom ROS2 applications into an existing Siemens Industrial Edge ecosystem. The developed Lifecycle Management Interface serves as a prototype for further development. An agile and modern web framework (VueJS) was used to develop the above mentioned Lifecycle Management Interface which allows for efficient addition of further functionalities to it.  
    \\
	
    The developed Lifecycle Management Interface will help and improve the overall stability of various mobile Robots developed using ROS2. It will further facilitate integration of the newly developed mobile robots into an existing ecosystem. It will also ensure cross compatibility of mobile robots and enable development of application in field of swarm robotics. The potential use cases of this Lifecycle Management Application will be discussed in the following chapter in detail.
    \\

    Another major part of this thesis is the Dockerisation of ROS2 based Applications. It provides major advantages in the development, deployment and operational cycle of a mobile robot. It is a step towards developing a CI/CD pipeline for the mobile robot development. With integration of Docker into the system, there is no need to develop a Operating System specific ROS Applications. With Docker multiple Applications with different Operating Systems and OS versions can run seamlessly on a single Computer system. It introduces a concept of microservices from web development into the ROS world. Which further simplifies the Network configuration and communication stability. A dockerized ROS2 application is then converted into Siemens Industrial Edge App. There is an existing ecosystem of IE Apps with an Appstore. After the conversion the ROS2 IE Apps can be downloaded from the IE Appstore and installed on a mobile robot without much hassle. 
    \\
    
    The Web interface for LifecycleManagement allows addition of multiple ROS2 nodes to the dashboard to be tracked and controlled. For a selected node the dashboard shows the current state of the node, available transitions and relevant log output to the console. From the dashboard it is possible to change the state of a lifecycle node to any other viable state. The automatic state transition has been implemented in the dashboard which only shows valid transitions.
    \\

    The development of ROS2-based IE Apps is still in the research and development phase and there is a lot of room for improvement. The frameworks and design patterns used for the developed application support further extension and improvement. Which should ensure compatibility with other applications developed in the future by the organization. The level of automation in the development process of the ROS2 Application is improved. Above all the viability, development process, and advantages of developing a ROS2 application with a managed lifecycle in a Docker environment are described in this thesis. 